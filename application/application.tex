\documentclass[]{book}
\usepackage{lmodern}
\usepackage{amssymb,amsmath}
\usepackage{ifxetex,ifluatex}
\usepackage{fixltx2e} % provides \textsubscript
\ifnum 0\ifxetex 1\fi\ifluatex 1\fi=0 % if pdftex
  \usepackage[T1]{fontenc}
  \usepackage[utf8]{inputenc}
\else % if luatex or xelatex
  \ifxetex
    \usepackage{mathspec}
  \else
    \usepackage{fontspec}
  \fi
  \defaultfontfeatures{Ligatures=TeX,Scale=MatchLowercase}
\fi
% use upquote if available, for straight quotes in verbatim environments
\IfFileExists{upquote.sty}{\usepackage{upquote}}{}
% use microtype if available
\IfFileExists{microtype.sty}{%
\usepackage{microtype}
\UseMicrotypeSet[protrusion]{basicmath} % disable protrusion for tt fonts
}{}
\usepackage{hyperref}
\hypersetup{unicode=true,
            pdftitle={Application for Tenure-Track Instructor Position in Statistics at UBC},
            pdfauthor={Vincenzo Coia},
            pdfborder={0 0 0},
            breaklinks=true}
\urlstyle{same}  % don't use monospace font for urls
\usepackage{natbib}
\bibliographystyle{apalike}
\usepackage{longtable,booktabs}
\usepackage{graphicx,grffile}
\makeatletter
\def\maxwidth{\ifdim\Gin@nat@width>\linewidth\linewidth\else\Gin@nat@width\fi}
\def\maxheight{\ifdim\Gin@nat@height>\textheight\textheight\else\Gin@nat@height\fi}
\makeatother
% Scale images if necessary, so that they will not overflow the page
% margins by default, and it is still possible to overwrite the defaults
% using explicit options in \includegraphics[width, height, ...]{}
\setkeys{Gin}{width=\maxwidth,height=\maxheight,keepaspectratio}
\IfFileExists{parskip.sty}{%
\usepackage{parskip}
}{% else
\setlength{\parindent}{0pt}
\setlength{\parskip}{6pt plus 2pt minus 1pt}
}
\setlength{\emergencystretch}{3em}  % prevent overfull lines
\providecommand{\tightlist}{%
  \setlength{\itemsep}{0pt}\setlength{\parskip}{0pt}}
\setcounter{secnumdepth}{5}
% Redefines (sub)paragraphs to behave more like sections
\ifx\paragraph\undefined\else
\let\oldparagraph\paragraph
\renewcommand{\paragraph}[1]{\oldparagraph{#1}\mbox{}}
\fi
\ifx\subparagraph\undefined\else
\let\oldsubparagraph\subparagraph
\renewcommand{\subparagraph}[1]{\oldsubparagraph{#1}\mbox{}}
\fi

%%% Use protect on footnotes to avoid problems with footnotes in titles
\let\rmarkdownfootnote\footnote%
\def\footnote{\protect\rmarkdownfootnote}

%%% Change title format to be more compact
\usepackage{titling}

% Create subtitle command for use in maketitle
\providecommand{\subtitle}[1]{
  \posttitle{
    \begin{center}\large#1\end{center}
    }
}

\setlength{\droptitle}{-2em}

  \title{Application for Tenure-Track Instructor Position in Statistics at UBC}
    \pretitle{\vspace{\droptitle}\centering\huge}
  \posttitle{\par}
    \author{Vincenzo Coia}
    \preauthor{\centering\large\emph}
  \postauthor{\par}
      \predate{\centering\large\emph}
  \postdate{\par}
    \date{2019-12-01}


\begin{document}
\maketitle

{
\setcounter{tocdepth}{1}
\tableofcontents
}
\hypertarget{preamble}{%
\chapter{Preamble}\label{preamble}}

\hypertarget{cover-letter}{%
\chapter{Cover Letter}\label{cover-letter}}

hyperlinks to one or two examples of my teaching materials. Note that I don't see any of these as being ``finished'', but rather a work-in-progress.

\begin{itemize}
\tightlist
\item
  For STAT 545A/547M (Exploratory Data Analysis), I made the \href{https://stat545guidebook.netlify.com/}{lecture notes} and \href{https://stat545.stat.ubc.ca/}{course website}. Less relevant are Lectures 11+, taught by Dr.~Firas Moosvi, and Lecture 6, a guest lecture taught by my TA, Victor Yuan.
\item
  For DSCI 551 (Probability for Data Science), I wrote the \href{https://ubc-mds.github.io/DSCI_551_stat-prob-dsci/lectures/}{lecture notes} by using Dr.~Mike Gelbart's notes from the previous year as scaffolding.
\item
  I'm writing an open-source book on regression analysis for data science, called \href{https://interpreting-regression.netlify.com/}{Interpreting Regression}. It's in its early stages.
\end{itemize}

the names of three references who have been asked to send reference letters:

\begin{enumerate}
\def\labelenumi{\arabic{enumi}.}
\tightlist
\item
  Dr.~Tiffany Timbers (current supervisor)
\item
  Dr.~Michael Gelbart (current supervisor)
\item
  Dr.~Harry Joe (PhD supervisor)
\end{enumerate}

\hypertarget{curriculum-vitae}{%
\chapter{Curriculum Vitae}\label{curriculum-vitae}}

\hypertarget{work-experience}{%
\section{Work Experience}\label{work-experience}}

2017/01 - present\\
\textbf{Lecturer of Data Science}\\
(Initially: Postdoctoral Teaching and Learning Fellow)\\
Masters of Data Science Program, and the Department of Statistics\\
The University of British Columbia\\
Vancouver, BC

2009/05 - 2014/05\\
\textbf{Short-term statistical consulting} (6 projects)\\
UBC and private

\hypertarget{education}{%
\section{Education}\label{education}}

\textbf{PhD in Statistics}\\
2012/09 - 2017/02\\
The University of British Columbia\\
Conferred May 29, 2017

\textbf{MSc in Mathematics and Statistics (Statistics)}\\
2011/09 - 2012/08\\
Brock University\\
Conferred on October 13, 2012

\textbf{BSc (3-year) in Biological Sciences}\\
Minor in Earth Sciences\\
2005/09 - 2011/04\\
Brock University\\
Conferred ``With Distinction'' on October 22, 2011

\textbf{BSc (Honours) Mathematics Integrated with Computers and Applications}\\
Concentration in Statistics\\
2005/09 - 2011/04\\
Brock University\\
Conferred ``With First-Class Standing'' on June 7, 2011

\hypertarget{volunteer-positions}{%
\subsection{Volunteer Positions}\label{volunteer-positions}}

2016/09 - 2016/02\\
\textbf{Science World at TELUS World of Science }\\
Vancouver, BC\\
78.15 hours

2013/10 - 2014/05\\
\textbf{Beaty Biodiversity Museum: Events Volunteer}\\
Vancouver, BC\\
35.0 hours

2013/04 - 2013/09\\
\textbf{UBC Farm}\\
Vancouver, BC\\
102.5 hours

2011/06 - 2011/08\\
\textbf{Project S.H.A.R.E. community garden}\\
Niagara Falls, ON\\
15.0 hours

\hypertarget{research-assistantships}{%
\subsection{Research Assistantships}\label{research-assistantships}}

2013/05 - 2013/08\\
\textbf{Robust penalized regression}\\
Supervisor: Dr.~Gabriela Cohen-Freue\\
Department of Statistics\\
The University of British Columbia\\
Vancouver, BC

2012/05 - 2012/08\\
2011/05 - 2011/08\\
2010/05 - 2010/08\\
\textbf{Extreme value modelling}\\
Supervisor: Dr.~Mei Ling Huang\\
Department of Mathematics\\
Brock University\\
St.~Catharines, ON

2010/09 - 2011/06\\
\textbf{Quantum monte carlo simulations}\\
Supervisors: Dr.~Stuart Rothstein; Dr.~Wai Kong (John) Yuen\\
Department of Chemistry and Department of Mathematics\\
Brock University\\
St.~Catharines, ON

\hypertarget{teaching-assistantships}{%
\subsection{Teaching Assistantships}\label{teaching-assistantships}}

\textbf{Duration}: From the latter part of my undergrad, to the end of my PhD.

UBC:

\begin{itemize}
\tightlist
\item
  \textbf{SCIE 300: Communicating Science} (5x)
\end{itemize}

Brock University:

\begin{itemize}
\tightlist
\item
  \textbf{MATH 4P82/5P82: Non-parametric Statistics}
\item
  \textbf{MATH 3P82: Regression Analysis}
\item
  \textbf{MATH 4P81/5P81: Sampling Theory}
\item
  \textbf{MATH 3P81: Experimental Design} (2x)
\item
  \textbf{MATH 2F40: Mathematics Integrated w/ Computers and Applications II}
\end{itemize}

\hypertarget{publications-and-talks}{%
\section{Publications and Talks}\label{publications-and-talks}}

\hypertarget{articles-submitted-to-refereed-journals}{%
\subsection{Articles Submitted to Refereed Journals}\label{articles-submitted-to-refereed-journals}}

\begin{itemize}
\item
  Huang, M.L., Coia, V., and Brill, P.H. (2013) A cluster truncated
  Pareto distribution and its applications. ISRN Probability and
  Statistics 2013: Article ID 265373.
\item
  Ayad, M., Coia, V., and Kihel, O. (2014) The number of relatively
  prime subsets of a finite union of sets of consecutive integers.
  Journal of Integer Sequences 17: Article 14.3.7
\item
  Coia, V., and Huang, M.L. (2014) A sieve model for extreme
  values. Journal of Statistical Computation and Simulation.
  84(8):16921710.
\end{itemize}

\hypertarget{articles-submitted-to-conference-proceedings}{%
\subsection{Articles Submitted to Conference Proceedings}\label{articles-submitted-to-conference-proceedings}}

\begin{itemize}
\tightlist
\item
  Huang, M. L., Coia, V., and Brill, P.H., A mixture truncated
  Pareto distribution, In JSM Proceedings 2012, Statistical Computing
  Section, Alexandria, VA: American Statistical Association, pp.

  \begin{enumerate}
  \def\labelenumi{\arabic{enumi}.}
  \setcounter{enumi}{24882497}
  \item
  \end{enumerate}
\end{itemize}

\hypertarget{conference-and-roundtable-contributions}{%
\subsection{Conference and Roundtable Contributions}\label{conference-and-roundtable-contributions}}

\begin{itemize}
\item
  Coia, V., Nolde, N., and Joe, H. Forecasting Extremes for Flooding
  (Invited Talk). The 44th Annual Meeting of the Statistical Society
  of Canada. May 29June 1, 2016 at Brock University, St.~Catharines,
  ON.
\item
  Coia, V., and Jeanniard du Dot, T. (Invited Demonstration) ``Using
  the Grammar of Graphics and Interactivity to explore Biologging Data
  in R''. May 6, 2015. Building a Bioanalytical Theory for Analysis of
  Marine Mammal Movements: A Peter Wall International Research
  Roundtable. The University of British Columbia, Vancouver, BC.
\item
  Coia, V. ``Flood Warning: An Application of High-Quantile Regression''
  (Contributed Talk). SFU/UBC Joint Graduate Student Seminar (Winter).
  February 28, 2015 at the SFU Harbour Centre, Vancouver, BC.
\item
  Coia, V. ``A New Sieve Model for Extreme Values'' (Contributed Talk).
  SFU/UBC Joint Graduate Student Seminar (Fall). September 29, 2012 at
  the SFU Harbour Centre, Vancouver, BC.
\item
  Coia, V., and Huang, M.L. ``On Estimation of Heavy Tailed
  Distributions'' (Contributed Talk). The 40th Annual Meeting of the
  Statistical Society of Canada. June 36, 2012 at the University of
  Guelph, Guelph, ON.
\item
  Huang, M.L., Coia, V., and Brill, P.H. ``A Mixture Truncated Pareto
  Distribution'' (Contributed Talk). The 2012 Joint Statistical
  Meetings. July 28August 2, 2012 at San Diego, California
\end{itemize}

\hypertarget{awards}{%
\section{Awards}\label{awards}}

\hypertarget{university-issued}{%
\subsection{University Issued}\label{university-issued}}

\begin{itemize}
\tightlist
\item
  2012/09 - 2016/08: Four-Year Fellowship
\item
  2012/09 - 2016/08: Faculty of Science Graduate Award
\item
  2011/09: Dean of Graduate Studies Excellence Scholarship
\item
  2011/06/07: Dean's Gold Medal
\item
  2011/06/07: Distinguished Undergraduate Student Award in Mathematics
\item
  2011/03: President's Surgite Award
\end{itemize}

\hypertarget{nationally-recognized}{%
\subsection{Nationally Recognized}\label{nationally-recognized}}

\begin{itemize}
\tightlist
\item
  2013/06: Governor General of Canada's Gold Medal
\item
  2012/09 - 2015/08: NSERC Postgraduate Award (Doctoral, 3-year)
\item
  2011/09 - 2012/08: NSERC Alexander Graham Bell Canada Graduate Scholarship (Masters)
\item
  2010/05 - 2010/08: NSERC Undergraduate Student Research Award
\end{itemize}

\hypertarget{individual-donors}{%
\subsection{Individual Donors}\label{individual-donors}}

\begin{itemize}
\tightlist
\item
  2012/04: Dr.~Jack Lightstone \& Dorothy Markiewicz Scholarship
\item
  2012/04: Dr.~Raymond \& Mrs.~Sachi Moriyama Grad. Fellowship
\item
  2012/03: Tomlinson Entrance Scholarship for Excellence in Mathematics and Science
\item
  2011/06: John and Roslyn Reed Book Prize
\item
  2010/09: Art Bicknell Scholarship in Mathematics
\item
  2010/09: Ian D. Beddis Family Scholarship
\item
  2010/09: Terry and Sue White Mathematics and Science Scholarship
\item
  2007/09: M.J. (``Mel'') Farquharson Scholarship
\item
  2006/09: Scholler Foundation Scholarship in Chemistry
\end{itemize}

\hypertarget{athletic-awards}{%
\subsection{Athletic Awards}\label{athletic-awards}}

\begin{itemize}
\tightlist
\item
  2016/04/03: Sportsmanship Award, UBC Thunderbirds Sport Clubs (Fencing)
\item
  2016/02/28: Bronze Medal in Senior Mixed Epee (Intercollegiate Tournament, UBC)
\item
  2015/11/16: Bronze Medal in Senior Mixed Epee (Remembrance Day Tournament, UBC)
\item
  2012/03/28: RM Davis Surgite Award
\item
  2009/12/12: First Place Award Winter Epeedemic, Toronto Fencing Club
\item
  2009/03: Varsity Fencing Rookie of the Year Award
\end{itemize}

\hypertarget{declined-awards}{%
\subsection{Declined Awards}\label{declined-awards}}

\begin{itemize}
\tightlist
\item
  2012/04: Ontario Graduate Scholarship (Doctoral)
\item
  2011/05: Ontario Graduate Scholarship (Masters)
\end{itemize}

\hypertarget{professional-activities}{%
\section{Professional Activities}\label{professional-activities}}

Dept. of Statistics, The University of British Columbia:

\begin{itemize}
\tightlist
\item
  2016/05 - 2016/06: \textbf{Search committee member (for CRC 2 faculty position)}
\item
  2015/04/30: \textbf{Organizer of the ``\href{http://stat.ubc.ca/~vincen.coia/abstractworkshop.html}{How to Write an Awesome Abstract}'' workshop}
\item
  2014/06 - 2015/05: \textbf{Organizer of the \href{http://stat.ubc.ca/~vincen.coia/seminar.html}{SFU/UBC Joint Seminar}}
\item
  2014/06 - 2014/09: \textbf{Organizer of the Graduate Student Trip}
\item
  2014/04 - 2015/05: \textbf{Co-founder of the Graduate Writing Forums}
\item
  2013/06 - 2014/05: \textbf{Statistics Graduate Student Representative}
\end{itemize}

\hypertarget{statement-of-vision-for-education-in-statistics-and-data-science}{%
\chapter{Statement of vision for education in statistics and data science}\label{statement-of-vision-for-education-in-statistics-and-data-science}}

\hypertarget{statement-of-teaching-and-training-philosophy}{%
\chapter{statement of teaching and training philosophy}\label{statement-of-teaching-and-training-philosophy}}

My teaching takes to heart what I learned from Greg Wilson, that teaching is much more about motivating students than it is knowledge transfer. Students have access to any information they want through the internet, but a classroom environment has the powerful advantage of the presence of peers and a knowledgeable instructor. I take a semi-spontaneous approach to teaching, as inspired by improv comedy -- I have a rough guideline that I abide to, but am not afraid to deviate from this where necessary, responding to how the class is feeling about a topic. I use various methods to keep constant awareness of how the class is feeling, through simple methods such as asking questions, to more modern active learning strategies such as think-pair-share, which leverages the knowledge of peers.

I believe in quality over quantity, and this is especially important for a program as time-sensitive as the Master of Data Science program at UBC that I currently teach in. Here, I focus on fundamental concepts, asking the question ``what must the students absolutely know by the end of this course?'' To help answer this, I look to fundamental concepts \emph{as they relate to applications}, not necessarily how they are developed in academia. I stick to these core concepts, and show students just how far they can go by exploring deeper concepts and data science methods -- again, going back to motivation over knowledge transfer.

I believe that teaching is far less effective when done ``in a vacuum'', as opposed to collaboratively with the input and feedback from peers. I'm lucky to be involved with open and communicative colleagues who can share their input, to build world-class courses in data science. And I'm happy to be on the other end as well, providing input and feedback to my colleagues.

\hypertarget{diversity-statement}{%
\chapter{Diversity Statement}\label{diversity-statement}}

To me, people deserve the same respect regardless of their identity. Any differential treatment is discrimination, and is problematic because it leaves people feeling disrespected, or worse, puts an impediment on one's life.

Our society has made some great progress on inclusivity (the opposite of discrimination). Racial and sexual discrimination are significantly less these days compared to several decades ago. But the battle is still not over, as is evidenced by graffitied rainbow sidewalks and defaced mosques.

Some discrimination is still rife. Gender discrimination is one such example, which is now starting to see some progress. But there are still other types of discrimination that are not yet well known. One such example that I feel is happening is discrimination against fathers.

Not only is it healthy for each individual to receive respectful and equal treatment, we have so much to gain by having a diverse group of people around us. There's strength in diversity.

The problem goes deeper and more elusive with the well-intentioned people who still discriminate unintentionally:

The problem does not necessarily lie with a discriminative response from an individual, if the response was unintentional. Despite the best of intentions, it's our environment and society that is responsible for crafting an automatic response from an individual. So, a good-intentioned individual that does a double-take after seeing me and my male partner holding hands should not beat themselves up over it -- it just means that the cumulative effect of their environment over time crafted such a response.

\hypertarget{my-experience-with-discrimination}{%
\section{My experience with discrimination}\label{my-experience-with-discrimination}}

As a member of the LGBTQ+ community, I continue to experience discrimination. Me and my male partner holding hands in public still forms a spectacle to many, some even stopping to watch us after we've walked by. Wedding vendors still referring to a ``bride'' when mentioning our wedding. Being invited to a wedding under the condition that my partner and I show no affection.

Things were worse in my adolescence, where homosexuality was ``discouraged'' in my environment, leaving me feeling socially out of place and fearful. Luckily, very few extended family members have a problem with my identity, and the rest embrace me.

Even though I'm a cis male, I'm quite passionate about gender issues, because they are largely not being embraced by our society, and because we're all affected by it (although transgendered and non-binary people are affected on an entirely different level).

To me, any type of brainwashing is deplorable, yet gender brainwashing is ubiquitous. You won't find a pink yoga mat in the men's section.

pink

Whistler bathroom

The solution to gender discrimination does not involve abolishing the notion of gender altogether, because gender has been proven to be important to humanity. Solution is about rather identifying a spectrum for which the extremes might be called ``masculine'' and ``feminine''.

Women's bathroom at 49th \textbar\textbar.

Fatherhood

\hypertarget{contribution}{%
\section{Contribution}\label{contribution}}

To me, contributing to inclusivity is about creating an inclusive environment, especially when it comes to making course content; calling out discrimination, even when the discrimination is unintentional; and being a role model for others by being comfortable about who I am.

\hypertarget{creating-an-inclusive-environment}{%
\subsection{Creating an inclusive environment}\label{creating-an-inclusive-environment}}

Involves:

\begin{itemize}
\tightlist
\item
  Changing our language to be less presumptious. This means saying things such as ``pregnant people'' instead of ``pregnant women'', referring to one's ``partner'' instead of saying ``wife'' or ``husband'' (assuming a gender), not referring to someone by their race if not relevant (such as ``I was talking to a Latino man the other day, \ldots{}'').
\item
  Posting online content that suggest inclusivity. For example, not using data that indicates gender as binary (because it's naive and ultimately untrue, even if gender is paraded as ``sex''), and not indicating female:male ratios and parading naive terms such as ``gender balanced'' (because that's naive too), but rather indicating percentage belonging to a minority group (such as ``percent self-identifying as female'').
\item
  Removing spaces that discriminate by gender -- this means a complete decoupling of bathrooms and gender (one should never have to say ``I'm not allowed in this room because I'm a man/woman''). UBC needs to first take low-hanging fruit by abolishing gender from single-person bathrooms (like we have in the stats dept), then focus on root issues of multi-person bathrooms (which at first seem gender-related, but are actually not), such as an overall lack of privacy.
\item
  Disassociating identities with career roles.
\item
  Including a ``covenant'' or ``code of conduct'' in collaborative (and student) projects.
\end{itemize}

\hypertarget{calling-out-discrimination}{%
\subsection{Calling out discrimination}\label{calling-out-discrimination}}

Means pointing out when non-inclusive language or behaviour is used, whether intentionally or not. Critically, this should be done with compassion, as opposed to accusation, because (1) people may not know any better, and (2) even if they do know better, this language can accidentally slip due to many years of belonging to a less-inclusive environment. Calling out discrimination can help educate people of the first type, as well as help re-program people of the second type. Examples include:

\begin{itemize}
\tightlist
\item
  Someone telling someone else that they're in the ``wrong bathroom''.
\item
  Example with MDS website, and bringing up issues in our academic meetings.
\end{itemize}

``Coming out'' tells others that I'm proud of who I am, but also indicates sensitivity to issues of gender and sexual identity.

\begin{itemize}
\tightlist
\item
  Posting my preferred personal pronouns (he/him/his) and a rainbow emoji online.
\item
  Posting a Positive Space sticker outside of my office.
\item
  Contributing my \href{https://lgbtstem.wordpress.com/2019/11/09/an-interview-with-vincenzo-coia/}{LGBTSTEM interview}.
\end{itemize}

As a result of this, I'm hoping that others can feel safer around me, and that I inspire others to have pride in who they are as well. I also hope that it brings awareness to those who are not familiar with LGBTQ+ issues -- for example, the more someone sees preferred personal pronouns being specified, the more likely they are to look up why more and more people are posting this and what this means.


\end{document}
