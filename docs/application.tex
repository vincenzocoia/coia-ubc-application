\documentclass[]{book}
\usepackage{lmodern}
\usepackage{amssymb,amsmath}
\usepackage{ifxetex,ifluatex}
\usepackage{fixltx2e} % provides \textsubscript
\ifnum 0\ifxetex 1\fi\ifluatex 1\fi=0 % if pdftex
  \usepackage[T1]{fontenc}
  \usepackage[utf8]{inputenc}
\else % if luatex or xelatex
  \ifxetex
    \usepackage{mathspec}
  \else
    \usepackage{fontspec}
  \fi
  \defaultfontfeatures{Ligatures=TeX,Scale=MatchLowercase}
\fi
% use upquote if available, for straight quotes in verbatim environments
\IfFileExists{upquote.sty}{\usepackage{upquote}}{}
% use microtype if available
\IfFileExists{microtype.sty}{%
\usepackage{microtype}
\UseMicrotypeSet[protrusion]{basicmath} % disable protrusion for tt fonts
}{}
\usepackage{hyperref}
\hypersetup{unicode=true,
            pdftitle={Application for Tenure-Track Instructor Position in Statistics at UBC},
            pdfauthor={Vincenzo Coia},
            pdfborder={0 0 0},
            breaklinks=true}
\urlstyle{same}  % don't use monospace font for urls
\usepackage{natbib}
\bibliographystyle{apalike}
\usepackage{longtable,booktabs}
\usepackage{graphicx,grffile}
\makeatletter
\def\maxwidth{\ifdim\Gin@nat@width>\linewidth\linewidth\else\Gin@nat@width\fi}
\def\maxheight{\ifdim\Gin@nat@height>\textheight\textheight\else\Gin@nat@height\fi}
\makeatother
% Scale images if necessary, so that they will not overflow the page
% margins by default, and it is still possible to overwrite the defaults
% using explicit options in \includegraphics[width, height, ...]{}
\setkeys{Gin}{width=\maxwidth,height=\maxheight,keepaspectratio}
\IfFileExists{parskip.sty}{%
\usepackage{parskip}
}{% else
\setlength{\parindent}{0pt}
\setlength{\parskip}{6pt plus 2pt minus 1pt}
}
\setlength{\emergencystretch}{3em}  % prevent overfull lines
\providecommand{\tightlist}{%
  \setlength{\itemsep}{0pt}\setlength{\parskip}{0pt}}
\setcounter{secnumdepth}{5}
% Redefines (sub)paragraphs to behave more like sections
\ifx\paragraph\undefined\else
\let\oldparagraph\paragraph
\renewcommand{\paragraph}[1]{\oldparagraph{#1}\mbox{}}
\fi
\ifx\subparagraph\undefined\else
\let\oldsubparagraph\subparagraph
\renewcommand{\subparagraph}[1]{\oldsubparagraph{#1}\mbox{}}
\fi

%%% Use protect on footnotes to avoid problems with footnotes in titles
\let\rmarkdownfootnote\footnote%
\def\footnote{\protect\rmarkdownfootnote}

%%% Change title format to be more compact
\usepackage{titling}

% Create subtitle command for use in maketitle
\providecommand{\subtitle}[1]{
  \posttitle{
    \begin{center}\large#1\end{center}
    }
}

\setlength{\droptitle}{-2em}

  \title{Application for Tenure-Track Instructor Position in Statistics at UBC}
    \pretitle{\vspace{\droptitle}\centering\huge}
  \posttitle{\par}
    \author{Vincenzo Coia}
    \preauthor{\centering\large\emph}
  \postauthor{\par}
      \predate{\centering\large\emph}
  \postdate{\par}
    \date{2019-12-07}


\begin{document}
\maketitle

{
\setcounter{tocdepth}{1}
\tableofcontents
}
\hypertarget{cover-letter}{%
\chapter{Cover Letter}\label{cover-letter}}

\textbf{NOTE: THIS APPLICATION IS A WORK-IN-PROGRESS}

Dear members of the search committee,

I am writing to enthusiastically apply for a tenure-track instructor position in Statistics (\href{https://www.stat.ubc.ca/three-tenure-track-instructor-positions-statistics-35876}{Job \#35876}). Ever since joining the department of Statistics full time in early 2017 to advance our Data Science initiatives, I've come to realize just how ideal of a fit the educational leadership stream is for me. My aim with this application is to indicate why this is so, as well as why the department and UBC (and external to UBC!) would benefit by promoting me to the educational leadership stream.

Ultimately, in this position, I envision advancing all branches of Data Science education at UBC, especially through Statistics for Data Science. I'm equally excited to develop the new Minor in Data Science program as I am to continue developing the Master of Data Science (MDS) program. In fact, I'm actually rather torn between the two: there is still much to develop with MDS, and I'm rather invested in the program; but the allure of developing a new Minor program is a new equally exciting challenge that I can make valuable contributions to. Regardless of my level of involvement in these, I intend to keep an open stream of communication between the two programs, along with STAT 545A/547M, so that data science education at UBC can overall grow as a cohesive whole, as opposed to competing pieces.

My vision for doing this

I've identified four pillars of my career, which I hope will lay the foundation for my vision going forward in the teaching stream, as well as convey fit on my end. They are, in order of culmination:

\begin{enumerate}
\def\labelenumi{\arabic{enumi}.}
\item
  \textbf{Application}: addressing real problems with data science.
\item
  \textbf{Research}: discovering better ways to do data science.
\item
  \textbf{Development}: turning new research into programs and tools, which are ideally accessible to the public.
\item
  \textbf{Teaching}: spreading data literacy and enthusiasm to the public.
\end{enumerate}

Lastly, I'd like to elaborate on how my skills will be an asset to the department.

Excitement and promise for starting up the minor program, helping MDS and STAT 545A/547M evolve along with it.

hyperlinks to one or two examples of my teaching materials. Note that I don't see any of these as being ``finished'', but rather a work-in-progress.

\begin{itemize}
\tightlist
\item
  For STAT 545A/547M (Exploratory Data Analysis), I made the \href{https://stat545guidebook.netlify.com/}{lecture notes} and \href{https://stat545.stat.ubc.ca/}{course website}. Less relevant are Lectures 11+, taught by Dr.~Firas Moosvi, and Lecture 6, a guest lecture taught by my TA, Victor Yuan.
\item
  For DSCI 551 (Probability for Data Science), I wrote the \href{https://ubc-mds.github.io/DSCI_551_stat-prob-dsci/lectures/}{lecture notes} by using Dr.~Mike Gelbart's notes from the previous year as scaffolding.
\item
  I'm writing an open-source book on regression analysis for data science, called \href{https://interpreting-regression.netlify.com/}{Interpreting Regression}. It's in its early stages.
\end{itemize}

What I'm comfortable teaching (useful in parallel to the minor program in DS)

Tools; my comfort level with CS. Although I don't have formal training in CS outside of perhaps a few undergraduate courses, my approach to learning these things is to identify tools and techniques that would be useful, and commit to building these tools. I prefer this approach over prescribed training such as online courses, because I find it more genuine as I encounter concepts as they become relevant. For example, I've been learning web hosting as I've been learning about Hugo, blogdown, and netlify to create websites like my homepage, course websites like STAT 545A/547M, and hosting ``books'' such as Interpreting Regression online. I've learned shell scripting back in my PhD, and continue to learn more, as I embrace GNU Make and workflow automation in general.

Teaching tools and methods

the names of three references who have been asked to send reference letters:

\begin{enumerate}
\def\labelenumi{\arabic{enumi}.}
\tightlist
\item
  Dr.~Tiffany Timbers (current supervisor)
\item
  Dr.~Michael Gelbart (current supervisor)
\item
  Dr.~Harry Joe (PhD supervisor)
\end{enumerate}

I am eager to begin the position on the indicated July 1, 2020 start date, but am ultimately flexible.

Enthusiastically yours,

Dr.~Vincenzo Coia

\hypertarget{curriculum-vitae}{%
\chapter{Curriculum Vitae}\label{curriculum-vitae}}

\hypertarget{work-experience}{%
\section{Work Experience}\label{work-experience}}

2017/01 - present\\
\textbf{Lecturer of Data Science}\\
(Initially: Postdoctoral Teaching and Learning Fellow)\\
Masters of Data Science Program, and the Department of Statistics\\
The University of British Columbia\\
Vancouver, BC

2009/05 - 2014/05\\
\textbf{Short-term statistical consulting} (6 projects)\\
UBC and private

\hypertarget{education}{%
\section{Education}\label{education}}

\textbf{PhD in Statistics}\\
2012/09 - 2017/02\\
The University of British Columbia\\
Conferred May 29, 2017

\textbf{MSc in Mathematics and Statistics (Statistics)}\\
2011/09 - 2012/08\\
Brock University\\
Conferred on October 13, 2012

\textbf{BSc (3-year) in Biological Sciences}\\
Minor in Earth Sciences\\
2005/09 - 2011/04\\
Brock University\\
Conferred ``With Distinction'' on October 22, 2011

\textbf{BSc (Honours) Mathematics Integrated with Computers and Applications}\\
Concentration in Statistics\\
2005/09 - 2011/04\\
Brock University\\
Conferred ``With First-Class Standing'' on June 7, 2011

\hypertarget{volunteer-positions}{%
\subsection{Volunteer Positions}\label{volunteer-positions}}

2016/09 - 2016/02\\
\textbf{Science World at TELUS World of Science }\\
Vancouver, BC\\
78.15 hours

2013/10 - 2014/05\\
\textbf{Beaty Biodiversity Museum: Events Volunteer}\\
Vancouver, BC\\
35.0 hours

2013/04 - 2013/09\\
\textbf{UBC Farm}\\
Vancouver, BC\\
102.5 hours

2011/06 - 2011/08\\
\textbf{Project S.H.A.R.E. community garden}\\
Niagara Falls, ON\\
15.0 hours

\hypertarget{research-assistantships}{%
\subsection{Research Assistantships}\label{research-assistantships}}

2013/05 - 2013/08\\
\textbf{Robust penalized regression}\\
Supervisor: Dr.~Gabriela Cohen-Freue\\
Department of Statistics\\
The University of British Columbia\\
Vancouver, BC

2012/05 - 2012/08\\
2011/05 - 2011/08\\
2010/05 - 2010/08\\
\textbf{Extreme value modelling}\\
Supervisor: Dr.~Mei Ling Huang\\
Department of Mathematics\\
Brock University\\
St.~Catharines, ON

2010/09 - 2011/06\\
\textbf{Quantum monte carlo simulations}\\
Supervisors: Dr.~Stuart Rothstein; Dr.~Wai Kong (John) Yuen\\
Department of Chemistry and Department of Mathematics\\
Brock University\\
St.~Catharines, ON

\hypertarget{teaching-assistantships}{%
\subsection{Teaching Assistantships}\label{teaching-assistantships}}

\textbf{Duration}: From the latter part of my undergrad, to the end of my PhD.

UBC:

\begin{itemize}
\tightlist
\item
  \textbf{SCIE 300: Communicating Science} (5x)
\end{itemize}

Brock University:

\begin{itemize}
\tightlist
\item
  \textbf{MATH 4P82/5P82: Non-parametric Statistics}
\item
  \textbf{MATH 3P82: Regression Analysis}
\item
  \textbf{MATH 4P81/5P81: Sampling Theory}
\item
  \textbf{MATH 3P81: Experimental Design} (2x)
\item
  \textbf{MATH 2F40: Mathematics Integrated w/ Computers and Applications II}
\end{itemize}

\hypertarget{publications-and-talks}{%
\section{Publications and Talks}\label{publications-and-talks}}

\hypertarget{articles-submitted-to-refereed-journals}{%
\subsection{Articles Submitted to Refereed Journals}\label{articles-submitted-to-refereed-journals}}

\begin{itemize}
\item
  Huang, M.L., Coia, V., and Brill, P.H. (2013) A cluster truncated
  Pareto distribution and its applications. ISRN Probability and
  Statistics 2013: Article ID 265373.
\item
  Ayad, M., Coia, V., and Kihel, O. (2014) The number of relatively
  prime subsets of a finite union of sets of consecutive integers.
  Journal of Integer Sequences 17: Article 14.3.7
\item
  Coia, V., and Huang, M.L. (2014) A sieve model for extreme
  values. Journal of Statistical Computation and Simulation.
  84(8):16921710.
\end{itemize}

\hypertarget{articles-submitted-to-conference-proceedings}{%
\subsection{Articles Submitted to Conference Proceedings}\label{articles-submitted-to-conference-proceedings}}

\begin{itemize}
\tightlist
\item
  Huang, M. L., Coia, V., and Brill, P.H., A mixture truncated
  Pareto distribution, In JSM Proceedings 2012, Statistical Computing
  Section, Alexandria, VA: American Statistical Association, pp.

  \begin{enumerate}
  \def\labelenumi{\arabic{enumi}.}
  \setcounter{enumi}{24882497}
  \item
  \end{enumerate}
\end{itemize}

\hypertarget{conference-and-roundtable-contributions}{%
\subsection{Conference and Roundtable Contributions}\label{conference-and-roundtable-contributions}}

\begin{itemize}
\item
  Coia, V., Nolde, N., and Joe, H. Forecasting Extremes for Flooding
  (Invited Talk). The 44th Annual Meeting of the Statistical Society
  of Canada. May 29June 1, 2016 at Brock University, St.~Catharines,
  ON.
\item
  Coia, V., and Jeanniard du Dot, T. (Invited Demonstration) ``Using
  the Grammar of Graphics and Interactivity to explore Biologging Data
  in R''. May 6, 2015. Building a Bioanalytical Theory for Analysis of
  Marine Mammal Movements: A Peter Wall International Research
  Roundtable. The University of British Columbia, Vancouver, BC.
\item
  Coia, V. ``Flood Warning: An Application of High-Quantile Regression''
  (Contributed Talk). SFU/UBC Joint Graduate Student Seminar (Winter).
  February 28, 2015 at the SFU Harbour Centre, Vancouver, BC.
\item
  Coia, V. ``A New Sieve Model for Extreme Values'' (Contributed Talk).
  SFU/UBC Joint Graduate Student Seminar (Fall). September 29, 2012 at
  the SFU Harbour Centre, Vancouver, BC.
\item
  Coia, V., and Huang, M.L. ``On Estimation of Heavy Tailed
  Distributions'' (Contributed Talk). The 40th Annual Meeting of the
  Statistical Society of Canada. June 36, 2012 at the University of
  Guelph, Guelph, ON.
\item
  Huang, M.L., Coia, V., and Brill, P.H. ``A Mixture Truncated Pareto
  Distribution'' (Contributed Talk). The 2012 Joint Statistical
  Meetings. July 28August 2, 2012 at San Diego, California
\end{itemize}

\hypertarget{awards}{%
\section{Awards}\label{awards}}

\hypertarget{university-issued}{%
\subsection{University Issued}\label{university-issued}}

\begin{itemize}
\tightlist
\item
  2012/09 - 2016/08: Four-Year Fellowship
\item
  2012/09 - 2016/08: Faculty of Science Graduate Award
\item
  2011/09: Dean of Graduate Studies Excellence Scholarship
\item
  2011/06/07: Dean's Gold Medal
\item
  2011/06/07: Distinguished Undergraduate Student Award in Mathematics
\item
  2011/03: President's Surgite Award
\end{itemize}

\hypertarget{nationally-recognized}{%
\subsection{Nationally Recognized}\label{nationally-recognized}}

\begin{itemize}
\tightlist
\item
  2013/06: Governor General of Canada's Gold Medal
\item
  2012/09 - 2015/08: NSERC Postgraduate Award (Doctoral, 3-year)
\item
  2011/09 - 2012/08: NSERC Alexander Graham Bell Canada Graduate Scholarship (Masters)
\item
  2010/05 - 2010/08: NSERC Undergraduate Student Research Award
\end{itemize}

\hypertarget{individual-donors}{%
\subsection{Individual Donors}\label{individual-donors}}

\begin{itemize}
\tightlist
\item
  2012/04: Dr.~Jack Lightstone \& Dorothy Markiewicz Scholarship
\item
  2012/04: Dr.~Raymond \& Mrs.~Sachi Moriyama Grad. Fellowship
\item
  2012/03: Tomlinson Entrance Scholarship for Excellence in Mathematics and Science
\item
  2011/06: John and Roslyn Reed Book Prize
\item
  2010/09: Art Bicknell Scholarship in Mathematics
\item
  2010/09: Ian D. Beddis Family Scholarship
\item
  2010/09: Terry and Sue White Mathematics and Science Scholarship
\item
  2007/09: M.J. (``Mel'') Farquharson Scholarship
\item
  2006/09: Scholler Foundation Scholarship in Chemistry
\end{itemize}

\hypertarget{athletic-awards}{%
\subsection{Athletic Awards}\label{athletic-awards}}

\begin{itemize}
\tightlist
\item
  2016/04/03: Sportsmanship Award, UBC Thunderbirds Sport Clubs (Fencing)
\item
  2016/02/28: Bronze Medal in Senior Mixed Epee (Intercollegiate Tournament, UBC)
\item
  2015/11/16: Bronze Medal in Senior Mixed Epee (Remembrance Day Tournament, UBC)
\item
  2012/03/28: RM Davis Surgite Award
\item
  2009/12/12: First Place Award Winter Epeedemic, Toronto Fencing Club
\item
  2009/03: Varsity Fencing Rookie of the Year Award
\end{itemize}

\hypertarget{declined-awards}{%
\subsection{Declined Awards}\label{declined-awards}}

\begin{itemize}
\tightlist
\item
  2012/04: Ontario Graduate Scholarship (Doctoral)
\item
  2011/05: Ontario Graduate Scholarship (Masters)
\end{itemize}

\hypertarget{professional-activities}{%
\section{Professional Activities}\label{professional-activities}}

Dept. of Statistics, The University of British Columbia:

\begin{itemize}
\tightlist
\item
  2016/05 - 2016/06: \textbf{Search committee member (for CRC 2 faculty position)}
\item
  2015/04/30: \textbf{Organizer of the ``\href{http://stat.ubc.ca/~vincen.coia/abstractworkshop.html}{How to Write an Awesome Abstract}'' workshop}
\item
  2014/06 - 2015/05: \textbf{Organizer of the \href{http://stat.ubc.ca/~vincen.coia/seminar.html}{SFU/UBC Joint Seminar}}
\item
  2014/06 - 2014/09: \textbf{Organizer of the Graduate Student Trip}
\item
  2014/04 - 2015/05: \textbf{Co-founder of the Graduate Writing Forums}
\item
  2013/06 - 2014/05: \textbf{Statistics Graduate Student Representative}
\end{itemize}

\hypertarget{statement-of-vision-for-education-in-statistics-and-data-science}{%
\chapter{Statement of vision for education in statistics and data science}\label{statement-of-vision-for-education-in-statistics-and-data-science}}

\hypertarget{vision}{%
\section{Vision}\label{vision}}

To me, innovation in the teaching of data science and statistics for data science involves:

\begin{itemize}
\tightlist
\item
  improving the delivery of a data science curriculum
\item
  improving the evaluation of students
\item
  redesigning the map of statistics from the perspective of data science to provide a framework for solving real-world problems

  \begin{itemize}
  \tightlist
  \item
    Interpreting Regression book (\textbf{EL})
  \end{itemize}
\item
  bringing to life underutilized tools and techniques.

  \begin{itemize}
  \tightlist
  \item
    distributional forecasting
  \item
    coperate for copulas:
  \item
    copulas
  \item
    extreme value modelling
  \end{itemize}
\item
  focussing on relevant real-world problems.
\item
  defining data science and maintaining a vision for our organizations (minor program; MDS; STAT 545A/547M, etc)
\item
  designing curricula for responsible use of data science.
\end{itemize}

There is a lot of research out there that's relevant for the effective teaching of data science. Resources like Greg Wilson's ``Teaching Tech Together'' or UBC's Instructional Skills Workshop provide a wealth of tools and techniques to be more effective. I'm all about keeping in touch with this community, and engaging in conversation and the sharing of ideas to help make strides in this area.

There seems to be ample attention placed on delivery and student evaluation by educational leaders. Where I intend to make my biggest mark is in figuring out how to apply Statistics to real-world problems. Statistical Science (importantly) is generally about setting up a framework of stochastic assumptions, and describing properties and methods under this framework. An example of a framework is linear regression; another, a Markov model. But the reality is that real examples don't neatly satisfy assumptions. Instead, some assumptions are better \emph{approximations} than others, yet others have different \emph{uses} than others. Thinking about assumptions in this respect brings about questions and ways of thinking that have traditionally not been adequately addressed in the literature nor the classroom, yet are defining for Statistics for Data Science -- hence why I think of this as ``research''. I call this approach a ``problem-first'' approach, as opposed to a ``method-first'' approach taken by Statistical Science. Even Applied Statistics may seem to take a ``problem-first'' approach, but the motivating real-world problem tends to be used as motivation for setting up a framework of assumptions, for which properties are then described.

I've been developing this way of thinking with MDS, but am nowhere near complete. I wish to continue this approach throughout my career.

Another area I intend to make my biggest contribution is on bringing to life underutilized tools and techniques that have already been described by statisticians. Specifically, I find that there is a lack of attention to the use of distributions in data science, perhaps stemming from their misunderstanding (for example, parametric functions ``needed'' as assumptions). There is a wealth of value that using both parametric and non-parametric distributions can bring to data science. To this extent, I am developing an R package that works with distributions as objects, as opposed to handling the unwieldly pieces of functions like \texttt{dnorm()} and \texttt{rnbinom()}. My goal is to submit this package to CRAN and ROpenSci, and perhaps grab the attention of RStudio and the \texttt{tidymodels} team. By extension, another area of statistics that I believe are underappreciated are copulas. Once distributional forecasting is truly appreciated, then copulas become invaluable for parametric modelling for prediction, as well as providing new insight into the properties of data. Extreme value theory also becomes invaluable for those cases that have heavy tailed data, such as at least one capstone project from Seahorse strategies (the team didn't end up using EVT because of lack of attention here, but should have), as well as organizations tackling the issue of flood forecasting and flood frequency assessments. I believe that copulas are currently not accessible enough due to complicated formulas and jargon, yet are tremendously useful for distributional forecasting. I hope to make this framework more accessible by developing an R package called \texttt{coperate} and \texttt{cmc} for easy manipulation of copulas, and easily created copula-based models (respectively). Again, submitted to CRAN and ROpenSci. Extreme value models will be easy to build using the distplyr package I mentioned earlier by easily grafting distributional tails with \texttt{distplyr::right\_connect()} and \texttt{distplyr::left\_connect()}.

This means continuing to work with external partners, such as the Master of Data Science (MDS) capstone partners, to understand their pressing data issues, and lend a hand once in a while. This is important for keeping myself and data science education at UBC relevant, as well as continually expanding my skillset. Besides having mentored MDS capstone projects for the past three years, I also did some work on flood forecasting with BGC Engineering this past summer.

Other ways I currently stay in touch with the data science community is through Twitter, the data science community at UBC (especially MDS's Slack channel), newsletters such as ROpenSci and RStudio, as well as just plain looking for solutions to a new problem, or new solutions to an old problem.

Examples of why ``Interpreting Regression'' is necessary:

\begin{itemize}
\tightlist
\item
  Survival analysis resources focus on the how of the Proportional hazards or Accelerated Failure models, or the survival curve estimation. First of all, with a lack of attention placed on the value and interpretation of distributions, survival curves and hazard functions are right off the bat a confusing introduction. Then, why we would even bother estimating such a thing, or setting up such a model, are not discussed. Also, the relationship between the Kaplan-Meier survival curve and the survival regression models is not clear. It's also not clear how one would apply machine learning methods with censored data. Again -- the framework has been set up (assumptions laid out for PH model or AF model), and properties described of these models (beta's are log hazard ratio), but this leaves students thinking ``I guess this is just what we do when faced with censored data'', as opposed to the more creative approach of making decisions on the modular components of modelling.
\item
  The teaching of MLE focusses on the how, and even then, it's hard to find a good resource on the how. But, after ample searching, I've yet to find the why of MLE, and this is because it's not framed in terms of deciding which assumptions to enforce in your model.
\item
  multiple imputation for missing data as plugging in a distribution to get marginal quantities. Without this being explained, students are left thinking ``I guess we should just use multiple imputation when faced with missing data, because that's just what we do''.
\end{itemize}

Envisioned impact outside the walls of the classroom by the time I become a Professor of Teaching: (EL)

\begin{itemize}
\tightlist
\item
  Hosting an annual ``Education in Data Science'' workshop/conference (we are already in the early stages of this).
\item
  Developing R packages with colleagues that make concepts easier to grasp:

  \begin{itemize}
  \tightlist
  \item
    distplyr for distributions: distributions as S3 objects in R, which can be manipulated and easily visualized.
  \item
    coperate for easily understanding and handling copulas.
  \end{itemize}
\item
  Writing public resources:

  \begin{itemize}
  \tightlist
  \item
    Book ``Interpreting Regression'' for reframing statistics for data science.
  \item
    Possibly another book, on distributional forecasting.
  \item
    Vignettes and journal articles about the use of said R packages.
  \item
    Course materials made public.
  \item
    Data science ``cheat sheets'', especially when it comes to a guide for making decision about model assumptions.
  \end{itemize}
\end{itemize}

\textbf{Problem-first organization}. I believe in quality over quantity, and this is especially important for a program as time-sensitive as the Master of Data Science program at UBC that I currently teach in. Here, I focus on fundamental concepts, asking the question ``what must the students absolutely know by the end of this course?'' To help answer this, I look to fundamental concepts \emph{as they relate to applications}, not necessarily how they are developed in academia. I stick to these core concepts, and show students just how far they can go by exploring deeper concepts and data science methods -- again, going back to motivation over knowledge transfer.
- {[} {]} problem-first approach when teaching data science
- {[} {]} Bringing in examples from Capstone for motivation
- Capstone. I believe that teaching is far less effective when done ``in a vacuum'', as opposed to collaboratively with the input and feedback from peers. I'm lucky to be involved with open and communicative colleagues who can share their input, to build world-class courses in data science. And I'm happy to be on the other end as well, providing input and feedback to my colleagues.

\hypertarget{skills-and-evidence}{%
\section{Skills and Evidence}\label{skills-and-evidence}}

\begin{itemize}
\item
  Existing examples:

  \begin{itemize}
  \tightlist
  \item
    Curriculum development of 551 throughout the years; 561 w/ Gaby and Sunny, later Tom; 531; 511?; 563; BAIT 509; 562; STAT 545A/547M.
  \item
    Arrangement of (mostly) statistical concepts as they relate to data science on a whole-program level.
  \end{itemize}
\item
  Promise: distplyr?
\item
  Promise: Promotion of distributional forecasting methods.
\item
  Promise: Host conference
\item
  teaching, mentorship and inspiration of colleagues: post-docs, Sunny, TA's for guest lecture.
\item
  Making resources public: 551, 545A/547M (expanded from Jenny's work).
\item
  Service to the academic profession, to the University, and to the community:

  \begin{itemize}
  \tightlist
  \item
    BGC project (important to stay relevant with pressing issues, and to bring industry experience to the program)
  \item
    MDS vision statement
  \item
    Blog post?
  \item
    ASDa workshops in 2017.
  \end{itemize}
\end{itemize}

{[} {]} Sharing the teaching: I like to give my TA's the opportunity to deliver a guest lecture. I'll guide them beforehand, and have a ``debrief'' session, usually a walk outside (inspired from ISW).

Maintaining a vision for our organization / MDS

\hypertarget{statement-of-teaching-and-training-philosophy}{%
\chapter{statement of teaching and training philosophy}\label{statement-of-teaching-and-training-philosophy}}

\hypertarget{teaching}{%
\section{Teaching}\label{teaching}}

You and I don't really know English. We only know enough of it to be a sufficiently useful tool for our own daily activities. I take this concept to heart as I design and deliver my courses. It's fallacious to design a course around \emph{topics}, but rather on building \emph{skills} for a certain purpose. To me, these skills formulated as a list of \textbf{learning objectives} make up the course's purpose, and so provides focus. For example, course content on the topic ``heteroskedasticity'' could be developed in many directions. The students, thinking they have to ``know heteroskedasticity'', get stuck studying its never-ending scope. Instead, a learning objective such as ``modify a linear model to appropriately incorporate heteroskedasticity based on data diagnostics'' provides direction for the teaching team, and clarity for the students. Clarity for the students equips them with a map of what they need to do to succeed. It's important to deliberately structure both course material and summative assessments towards these learning objectives, like a compass pointing north.

Speaking of structure, too much of it can stymie your course if taken overboard or held too tightly. This is because, when preparing material, you're not equipped with the clarity of how to most appropriately deliver material until it has been delivered. For example, a question for the class might have seemed insightful when you prepared it, but maybe during class the students aren't confused by it at all. Or, maybe you've underestimated the time it will take to deliver some material. The answer to addressing this issue is \textbf{responsive teaching}.

Responsive teaching to me is about staying in touch with the class to get a feel for areas that need more or less attention. For example, it means: taking a break when the energy of the class is low; responding to class confusion by explaining something in a different way on the whiteboard; or spending more or less time on something after realizing its importance mid-class. In this way, teaching is more akin to improv comedy, where actors must respond to random cues from the audience. Addressing this need in teaching requires letting go and trusting in yourself to respond appropriately in the moment, but also to be humble when you don't know something (which in fact builds trust). Spontaneity sometimes also requires realizing and appropriately acting on your agency in adapting a course. Understanding that a course is flexible allows you to be nimble during contact hours with students, as opposed to feeling stuck in delivering the course exactly as it was originally laid out. It's important to try to make changes for what's to come as opposed to what's already passed, but we can't always, in which case it's important to inform the students of any changes and why. I like doing so at the start of class while I give an overview of where we are in the course and where we're going.

I use various methods to keep constant awareness of how the class is feeling and connecting with the students. I use simple methods such as asking questions, to more modern active learning strategies such as think-pair-share, which leverages the knowledge of peers. Also, at the start of each class, I like to check in to see how things are going. I'll ask questions like ``how are we feeling about the quiz coming up next week?'', or ``how are we feeling this week?'', which tend to be successful. I no longer hold my office hours in my office, because the office hour model of students dropping by to ask a question and then leave is just not effective. There ends up being a queue of students, usually asking common questions, and these students feel pressured to leave so that others can get a turn. Instead, I hold my office hours in a lab-like room suitable for collaboration. I end up leading a whole-group discussion prompted by student questions. This also gives me even more insight as to how things are going in class, and allows me the opportunity to modify the course moving forward. I also like to keep an eye on the course Slack channel to provide more insight, as well as talking to students during the mid-class break or after class about questions they may have. During class, I like to pause to check for insight in the class.

When it comes to my actual delivery in class, my teaching takes to heart what I learned from Greg Wilson, that teaching is much more about motivating students than it is knowledge transfer. Students have access to any information they want through the internet, but a classroom environment has the powerful advantage of the presence of peers and a knowledgeable instructor. On the teaching front, this means being enthusiastic (it means something different for structuring the course, as I elaborate on in my vision for education in statistics). I'm pleased to that I consistently get feedback from students and peers that my enthusiasm is contagious in the classroom. I do love the material I teach. \textbf{Active learning techniques}: parsons problems, think-pair-share, live coding.

I believe the bigger theme of providing clarity about the course empowers the students, because they know what they have to do to succeed. As mentioned, providing learning objectives and their importance in summative assessments is part of providing clarity, but so is providing a detailed course syllabus and discussing it at the start of the course.

Improving my teaching involves \textbf{community engagement} and \textbf{after-action reviews}:

\textbf{Community}: stay in touch through weekly academic meetings, lunches, and slack; giving and receiving peer feedback on their teaching is also useful for improving.

An after-action review is a reflection after delivering material. As they say, hindsight is 20-20. Capturing this insight is a very effective way of improving a course, either for next year or mid-delivery. There are three ways I engage in an after-action review. First, I capture regular insight throughout the delivery of the course as GitHub Issues, so that they can easily be referred to in the future and by any of my colleagues. Secondly, I find keeping a teaching journal that's not tied to a specific course is useful for becoming a better teacher in general. Entries here are less frequent, and usually higher level. Thirdly, my colleagues and I engage in a ``retreat'' at the end of each term, to discuss our insight on our courses and the MDS program as a whole. All three of these methods are powerful ways to improve.

One thing I struggle with are student names. I feel embarrassed to use names because I'm sure to make mistakes, some probably quite embarrassing. But calling students by their name is important because it shows that the instructor cares. I already review class lists, but intend to step outside of my comfort zone by no longer avoiding using names.

Another thing I struggle with is writing in an orderly fashion on the whiteboard. I use the whiteboard for spontaneity / responsive teaching, which means I don't know how I'll ultimately be using the board throughout the class. I end up running out of space, and end up using the board in a non-linear fashion, which can sometimes be confusing. For now, I intend to remain mindful of this and keep practicing.

In summary, to me, teaching is about setting a clear purpose of the course through learning objectives, which should guide all aspects of the course. Making way for student success requires:

\begin{itemize}
\tightlist
\item
  responding appropriately to their needs by embracing some spontaneity in the classroom, being humble, and sometimes modifying the course as appropriate.
\item
  keeping an open dialogue with the students.
\item
  providing motivation and enthusiasm instead of knowledge transfer.
\end{itemize}

Improving my own teaching involves reflecting in an after-action review, and staying engaged with my colleagues who are also teaching.

\hypertarget{training}{%
\section{Training}\label{training}}

\begin{itemize}
\tightlist
\item
  Respecting people's time: giving subordinates the option of meeting electronically if they choose.
\item
  Making clear everyone's role in a team project (example: 561 project, capstone projects)
\item
  and getting a sense of the amount of time/involvement each person will be contributing.
\item
  Delegation levels, and being open to letting go (sometimes it's difficult because we want to take ownership)

  \begin{itemize}
  \tightlist
  \item
    Also, I believe that we don't delegate enough. As a result, we end up sacrificing valuable time where instead we could be really moving the needle in our program.
  \end{itemize}
\item
  Staying mindful of the potential of subconscious discrimination coming about by showing preference of some team members over others, whether it be race, gender, personality, skillset, etc. This means being deliberate about where I spend my attention.
\item
  Feedback: taking the time to provide feedback for one's work. I believe this shows respect for the other person, that they are worth my time, and provides an impetus for them to improve. Example: 561 labs 3 and 4 with Tom; detailed feedback on pieces of writing submitted by students (capstone).
\item
  I try to be genuine so that they can see that I'm also a human that makes mistakes and has gaps in my skillset, and lead a regular life outside of UBC. I think this also helps develop a stream of open communication for questions, concerns, and how things are going. Creating a culture that's safe for dissent, because my subordinates have skills, knowledge, and a viewpoint that I don't have, and from which I can learn from them.
\item
  One thing I need to improve upon for this coming capstone, as reflected in my evaluations, is to be more mindful of group dynamics. Even if things between me and each team member/student are fine, relationships between other team members/students might not be. This year, I'll be meeting with each student individually to see how things are going.
\end{itemize}

\hypertarget{diversity-statement}{%
\chapter{Diversity Statement}\label{diversity-statement}}

To me, people deserve the same respect regardless of their identity. Any differential treatment is discrimination, and is problematic because it leaves people feeling disrespected, or worse, puts an impediment on one's life.

Our society has made some great progress on inclusivity (the opposite of discrimination). Racial and sexual discrimination are significantly less these days compared to several decades ago. But the battle is still not over, as is evidenced by graffitied rainbow sidewalks and defaced mosques.

Some discrimination is still rife. Gender discrimination is one such example, which is now starting to see some progress. But there are still other types of discrimination that are not yet well known. One such example that I feel is happening is discrimination against fathers.

Not only is it healthy for each individual to receive respectful and equal treatment, we have so much to gain by having a diverse group of people around us. There's strength in diversity.

The problem goes deeper and more elusive with the well-intentioned people who still discriminate unintentionally:

The problem does not necessarily lie with a discriminative response from an individual, if the response was unintentional. Despite the best of intentions, it's our environment and society that is responsible for crafting an automatic response from an individual. So, a good-intentioned individual that does a double-take after seeing me and my male partner holding hands should not beat themselves up over it -- it just means that the cumulative effect of their environment over time crafted such a response.

In the classroom, ensuring students all feel welcome, and feel that they can come talk to me, or participate in class. Fairness. Being mindful about whether I may be paying less attention to certain groups, for example, on Slack. I was once accused of spending less energy on non-native English speakers on Slack. Although it's hard for us to recognize unconscious discrimination, I'm not convinced their claim was true -- but regardless, this experience taught me that I need to be more deliberate about how I respond to students. I like to think, ``how would I respond if this question or comment was made by someone else?''. I think the usefulness of this litmus test extends beyond Slack to general interactions with students.

\hypertarget{my-experience-with-discrimination}{%
\section{My experience with discrimination}\label{my-experience-with-discrimination}}

As a member of the LGBTQ+ community, I continue to experience discrimination. Me and my male partner holding hands in public still forms a spectacle to many, some even stopping to watch us after we've walked by. Wedding vendors still referring to a ``bride'' when mentioning our wedding. Being invited to a wedding under the condition that my partner and I show no affection.

Things were worse in my adolescence, where homosexuality was ``discouraged'' in my environment, leaving me feeling socially out of place and fearful. Luckily, very few extended family members have a problem with my identity, and the rest embrace me.

Even though I'm a cis male, I'm quite passionate about gender issues, because they are largely not being embraced by our society, and because we're all affected by it (although transgendered and non-binary people are affected on an entirely different level).

To me, any type of brainwashing is deplorable, yet gender brainwashing is ubiquitous. You won't find a pink yoga mat in the men's section.

pink

Whistler bathroom

The solution to gender discrimination does not involve abolishing the notion of gender altogether, because gender has been proven to be important to humanity. Solution is about rather identifying a spectrum for which the extremes might be called ``masculine'' and ``feminine''.

Women's bathroom at 49th \textbar\textbar.

Fatherhood. Nice article on the bias: \url{https://goodmenproject.com/families/mom-bias-in-the-parenting-community-why-is-there-no-discussion-about-it-wcz/}
Nice article on involved dads reaching similar levels of oxytocin with involved moms: \url{https://www.nbcnews.com/sciencemain/your-brain-fatherhood-dads-experience-hormonal-changes-too-research-shows-6C10333109}, which has one line that stands out when it comes to bias:

\begin{quote}
Rilling said the study of the fatherhood effect is a ``wide open frontier.''
\end{quote}

As you may know, my partner and I recently had a child through surrogacy. But, on the delivery day, the hospital was not prepared to handle a situation with two dads. It's important for the parents to form a bond with their child, and to help the surrogate avoid attachment issues with the child -- but the (well-intentioned) hospital staff overlooked both of these aspects. We (the parents) were supposed to be brought in as soon as the baby was born, for initial skin-to-skin contact, but the hospital staff didn't bring us in until an hour after birth. Furthermore, they gave our surrogate first skin-to-skin contact -- this promotes oxytocin in the surrogate, making it more difficult to part with the child, and denies the parents the opportunity for initial bonding. Worse, hospital staff kept referring to our surrogate as ``mom'', and referred to our son by her last name. Not only is this disrespectful for the parents, but it makes separation even more difficult for the surrogate.

In fact, there is an overarching issue of dads not being regarded as equally important figures as moms. Worse, this issue gets almost no attention. Moms are paraded as the caregivers, whereas dads are just there for support and don't deserve much say in parenthood. Just about anything to do with parenthood refers to mom -- for example:

\begin{itemize}
\tightlist
\item
  we could not find a book about parenthood that does not emphasize mom's role;
\item
  parenthood groups emphasize women over men, such as Facebook groups and community centers.
\item
  marketing. Companies like 4Moms is akin to a business supply brand being called ``4Businessmen''.
\item
  Even scientific research
\end{itemize}

Since this hasn't been socially recognized as an issue yet, in mentioning it, I risk sounding androcentric. This cannot be further from the truth. I recognize that women are far more marginalized than men are, and I fully embrace empowerment of women. Groups like R Ladies need support; I'm proud of the fact that we pay close attention to the amount of women we accept in the MDS program; and I'm no stranger to calling out

en though men are usually the priveledged

\hypertarget{contribution}{%
\section{Contribution}\label{contribution}}

To me, contributing to inclusivity is about creating an inclusive environment, especially when it comes to making course content; calling out discrimination, even when the discrimination is unintentional; and being a role model for others by being comfortable about who I am.

\hypertarget{creating-an-inclusive-environment}{%
\subsection{Creating an inclusive environment}\label{creating-an-inclusive-environment}}

Involves:

\begin{itemize}
\tightlist
\item
  Changing our language to be less presumptious. This means saying things such as ``pregnant people'' instead of ``pregnant women'', referring to one's ``partner'' instead of saying ``wife'' or ``husband'' (assuming a gender), not referring to someone by their race if not relevant (such as ``I was talking to a Latino man the other day, \ldots{}'').
\item
  Posting online content that suggest inclusivity. For example, not using data that indicates gender as binary (because it's naive and ultimately untrue, even if gender is paraded as ``sex''), and not indicating female:male ratios and parading naive terms such as ``gender balanced'' (because that's naive too), but rather indicating percentage belonging to a minority group (such as ``percent self-identifying as female'').
\item
  Removing spaces that discriminate by gender -- this means a complete decoupling of bathrooms and gender (one should never have to say ``I'm not allowed in this room because I'm a man/woman''). UBC needs to first take low-hanging fruit by abolishing gender from single-person bathrooms (like we have in the stats dept), then focus on root issues of multi-person bathrooms (which at first seem gender-related, but are actually not), such as an overall lack of privacy.
\item
  Disassociating identities with career roles.
\item
  Including a ``covenant'' or ``code of conduct'' in collaborative (and student) projects.
\end{itemize}

\hypertarget{calling-out-discrimination}{%
\subsection{Calling out discrimination}\label{calling-out-discrimination}}

Means pointing out when non-inclusive language or behaviour is used, whether intentionally or not. Critically, this should be done with compassion, as opposed to accusation, because (1) people may not know any better, and (2) even if they do know better, this language can accidentally slip due to many years of belonging to a less-inclusive environment. Calling out discrimination can help educate people of the first type, as well as help re-program people of the second type. Examples include:

\begin{itemize}
\tightlist
\item
  Someone telling someone else that they're in the ``wrong bathroom''.
\item
  Example with MDS website, and bringing up issues in our academic meetings.
\end{itemize}

``Coming out'' tells others that I'm proud of who I am, but also indicates sensitivity to issues of gender and sexual identity.

\begin{itemize}
\tightlist
\item
  Posting my preferred personal pronouns (he/him/his) and a rainbow emoji online.
\item
  Posting a Positive Space sticker outside of my office.
\item
  Contributing my \href{https://lgbtstem.wordpress.com/2019/11/09/an-interview-with-vincenzo-coia/}{LGBTSTEM interview}.
\end{itemize}

As a result of this, I'm hoping that others can feel safer around me, and that I inspire others to have pride in who they are as well. I also hope that it brings awareness to those who are not familiar with LGBTQ+ issues -- for example, the more someone sees preferred personal pronouns being specified, the more likely they are to look up why more and more people are posting this and what this means.

{[} {]} sensitivity to marginalized groups; embracing diversity.


\end{document}
